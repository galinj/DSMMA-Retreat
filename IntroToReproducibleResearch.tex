% Options for packages loaded elsewhere
\PassOptionsToPackage{unicode}{hyperref}
\PassOptionsToPackage{hyphens}{url}
%
\documentclass[
  ignorenonframetext,
]{beamer}
\usepackage{pgfpages}
\setbeamertemplate{caption}[numbered]
\setbeamertemplate{caption label separator}{: }
\setbeamercolor{caption name}{fg=normal text.fg}
\beamertemplatenavigationsymbolsempty
% Prevent slide breaks in the middle of a paragraph
\widowpenalties 1 10000
\raggedbottom
\setbeamertemplate{part page}{
  \centering
  \begin{beamercolorbox}[sep=16pt,center]{part title}
    \usebeamerfont{part title}\insertpart\par
  \end{beamercolorbox}
}
\setbeamertemplate{section page}{
  \centering
  \begin{beamercolorbox}[sep=12pt,center]{part title}
    \usebeamerfont{section title}\insertsection\par
  \end{beamercolorbox}
}
\setbeamertemplate{subsection page}{
  \centering
  \begin{beamercolorbox}[sep=8pt,center]{part title}
    \usebeamerfont{subsection title}\insertsubsection\par
  \end{beamercolorbox}
}
\AtBeginPart{
  \frame{\partpage}
}
\AtBeginSection{
  \ifbibliography
  \else
    \frame{\sectionpage}
  \fi
}
\AtBeginSubsection{
  \frame{\subsectionpage}
}
\usepackage{lmodern}
\usepackage{amssymb,amsmath}
\usepackage{ifxetex,ifluatex}
\ifnum 0\ifxetex 1\fi\ifluatex 1\fi=0 % if pdftex
  \usepackage[T1]{fontenc}
  \usepackage[utf8]{inputenc}
  \usepackage{textcomp} % provide euro and other symbols
\else % if luatex or xetex
  \usepackage{unicode-math}
  \defaultfontfeatures{Scale=MatchLowercase}
  \defaultfontfeatures[\rmfamily]{Ligatures=TeX,Scale=1}
\fi
% Use upquote if available, for straight quotes in verbatim environments
\IfFileExists{upquote.sty}{\usepackage{upquote}}{}
\IfFileExists{microtype.sty}{% use microtype if available
  \usepackage[]{microtype}
  \UseMicrotypeSet[protrusion]{basicmath} % disable protrusion for tt fonts
}{}
\makeatletter
\@ifundefined{KOMAClassName}{% if non-KOMA class
  \IfFileExists{parskip.sty}{%
    \usepackage{parskip}
  }{% else
    \setlength{\parindent}{0pt}
    \setlength{\parskip}{6pt plus 2pt minus 1pt}}
}{% if KOMA class
  \KOMAoptions{parskip=half}}
\makeatother
\usepackage{xcolor}
\IfFileExists{xurl.sty}{\usepackage{xurl}}{} % add URL line breaks if available
\IfFileExists{bookmark.sty}{\usepackage{bookmark}}{\usepackage{hyperref}}
\hypersetup{
  pdftitle={Introduction to Reproducibility},
  pdfauthor={Galin Jones},
  hidelinks,
  pdfcreator={LaTeX via pandoc}}
\urlstyle{same} % disable monospaced font for URLs
\newif\ifbibliography
\usepackage{graphicx,grffile}
\makeatletter
\def\maxwidth{\ifdim\Gin@nat@width>\linewidth\linewidth\else\Gin@nat@width\fi}
\def\maxheight{\ifdim\Gin@nat@height>\textheight\textheight\else\Gin@nat@height\fi}
\makeatother
% Scale images if necessary, so that they will not overflow the page
% margins by default, and it is still possible to overwrite the defaults
% using explicit options in \includegraphics[width, height, ...]{}
\setkeys{Gin}{width=\maxwidth,height=\maxheight,keepaspectratio}
% Set default figure placement to htbp
\makeatletter
\def\fps@figure{htbp}
\makeatother
\setlength{\emergencystretch}{3em} % prevent overfull lines
\providecommand{\tightlist}{%
  \setlength{\itemsep}{0pt}\setlength{\parskip}{0pt}}
\setcounter{secnumdepth}{-\maxdimen} % remove section numbering

\title{Introduction to Reproducibility}
\author{Galin Jones}
\date{DSMMA Retreat 2020}
\institute{School of Statistics\\
University of Minnesota\\
\href{mailto:galin@umn.edu}{\nolinkurl{galin@umn.edu}}\\
@JonesGalin}

\begin{document}
\frame{\titlepage}

\begin{frame}{Reproducibility}
\protect\hypertarget{reproducibility}{}

\begin{figure}
\centering
\includegraphics{~/Documents/Why.png}
\caption{Your work doesn't count if it can't be reproduced.}
\end{figure}

\end{frame}

\begin{frame}{Reproducibility}
\protect\hypertarget{reproducibility-1}{}

\begin{figure}
\centering
\includegraphics{~/Documents/OSF.png}
\caption{Open science--share everything}
\end{figure}

\end{frame}

\begin{frame}{Reproducibility}
\protect\hypertarget{reproducibility-2}{}

\begin{figure}
\centering
\includegraphics{~/Documents/Repro.png}
\caption{Everyone is welcome to participate.}
\end{figure}

\end{frame}

\begin{frame}{Computational Reproducibility}
\protect\hypertarget{computational-reproducibility}{}

A hallmark of the modern data science environment (industry and
academia) is sharing code and data. Make everything available that is
required to reproduce your computational results \emph{exactly}.

\bigskip

It begins by writing your code so that it is easily read by someone else
(future you for example). Free advice:

\begin{itemize}
\item[] Document your code.
\item[] Use descriptive variable names.
\item[] Format it to be easily read.
\item[] Order functions for linear reading.
\item[] Be consistent.
\end{itemize}

\end{frame}

\begin{frame}{Computational Reproducibility}
\protect\hypertarget{computational-reproducibility-1}{}

Reproducible data analysis and version control

\begin{itemize}
\item[] Git/GitHub
\item[] Emacs/RStudio/Spyder, 
\end{itemize}

Reproducible data

\begin{itemize}
\item[] Data repositories
\item[] Dataverse
\end{itemize}

Reproducible dynamic report generation

\begin{itemize}
\item[] R markdown/R Notebook/Jupyter/Pandoc
\end{itemize}

\end{frame}

\begin{frame}{Further Reading}
\protect\hypertarget{further-reading}{}

\begin{itemize}
\item[] R Open Science http://ropensci.github.io/reproducibility-guide/
\item[] Dataverse https://dataverse.org/
\item[] Code and Data for the Social Sciences:
A Practitioner’s Guide http://web.stanford.edu/~gentzkow/research/CodeAndData.pdf
\end{itemize}

\end{frame}

\end{document}
